\chapter{习题 10: 字符串数组,循环}

你可以用多种数据类型组成一个数组,要记得“字符串”和“字节数组”就是一个东西。
下一步就是将字符串放到数组里。这里还会为你介绍第一个循环结构—— \ident{for-loop}
(for 循环)——它会帮助你打印出这个新的数据结构。

有个有趣的地方是,其实已经有个字符串数组在你的程序中躲藏了一段时间了,那就是在
\ident{main} 函数参数里的 \ident{char *argv[]} 。下面一段代码会
打印出你通过命令行传递的所有参数:

\begin{code}{ex10.c}
<< d['code/ex10.c|pyg|l'] >>
\end{code}

\ident{for-loop} 的格式是这样的:

\begin{Verbatim}
    for(INITIALIZER; TEST; INCREMENTER) {
        CODE;
    }
\end{Verbatim}

接下来讲讲 \ident{for-loop} 的工作原理:

\begin{enumerate}
\item \ident{INITIALIZER} 是完成循环前的准备工作的代码,在本例中就是 \verb|i = 0| 
这句。
\item 接下来运行的是 \ident{TEST} 这个布尔表达式的检查动作,如果它的值是 false (0),
那么接下来就什么都不做了。
\item 运行代码 \ident{CODE},完成它要做的事情。
\item \ident{CODE} 运行之后,\ident{INCREMENTER} 部分就会被运行,
通常都是将某个变量递增,比如本例中的 \verb|i++| 。
\item 然后代码会从第 2 步开始重新运行, 直到 \ident{TEST} 的值成为 false (0) 为止。
\end{enumerate}

这个 \ident{for-loop} 将使用 \ident{argc} 和 \ident{argv} 
这两个变量来对命令行参数做如下处理:

\begin{enumerate}
\item 操作系统将每一条命令行参数当做字符串传递到数组 \ident{argv} 中,程序的名称
(./ex10)的位置是 0,其它的参数向后依次排列。
\item 操作系统还将 \ident{argc} 设为 \ident{argv} 中参数的个数,这样你处理它的时候
就不必担心超过最后一个参数了。
\item \ident{for-loop} 做了预备工作,设定 \verb|i = 1| 。
\item 之后它通过 \verb|i < argc| 这句判断 \ident{i} 是否小于 \ident{argc}。
一开始 $0 < 1$,所以判断通过。
\item 然后它接着运行代码,打印出 \ident{i} ,然后使用 \ident{i} 作为 \ident{argv}
 数组内部的索引。
\item 接着递增器将通过语法 \verb|i++| 运行。这算是 \verb|i = i + 1| 的简写方式。
\item 然后它就会重复上述代码,直到 \verb|i < argc| 的值成为 false (0),然后循环就结束了,余下的程序将继续运行。
\end{enumerate}

\section{你应该看到的结果}

跟这个程序打交道你需要用两种方式运行它。第一种是传递一些命令行参数从而赋值给 \ident{argc} 和 \ident{argv} 。第二种是运行时不传递任何参数,这样的话由于 \verb|i < argc|
一开始就是 false,\ident{for-loop} 将不会运行。

\begin{code}{ex10 的输出}
\begin{lstlisting}
<< d['code/ex10.out'] >>
\end{lstlisting}
\end{code}

\subsection{理解字符串数组}

从习题中你应该已经知道了 C 语言中“字符串数组”是怎样构造出来的了——通过结合
 \verb|char *str = "blah"| 和 \verb|char str[] = {'b','l','a','h'}| 
 来构建一个二维数组。第 14 行的 \verb|char *states[] = {...}| 就是这个二维组合的动作,
 其中一维是以字符串为元素,另一维度则是以构成字符串的字符作为元素。

头晕了吧?很多人都没有想过多重维度的概念,所以你应该试着在纸上构建出这个字符串数组:

\begin{enumerate}
\item 在这张纸上做一个矩形网格,最左列填入每一个 \emph{string(字符串)} 的索引值。
\item 最上面一行填入每一个 \emph{character(字符)} 的索引值。
\item 然后将字符串填入表格中间的空位,每个字符占一格。
\item 做好这个表格后,用它来手动追踪一遍代码。
\end{enumerate}

另一个弄明白的方法是通过使用你更熟悉的语言 (比如 Python 或者 Ruby) 建造一个相同的结构。

\section{让程序出错}

\begin{enumerate}
\item 选一个你喜欢的别种语言,用它来运行这个程序,但是要给它尽可能多的命令行参数。看看你能不能
通过给出太多的形参把这程序给搞崩了。
\item 将 \ident{i} 初始化为 0 看看会发生什么。你需要同时调整 \ident{argc} 吗?还是说它可以直接运行?为什么这里可以用基于 0 的索引方式呢?
\item 把 \ident{num\_states} 设置得更高导致错误,看看会发生什么。
\end{enumerate}

\section{加分习题}

\begin{enumerate}
\item 弄明白什么样的代码你可以放到 \ident{for-loop} 中。
\item 查找一下应该如何在 \ident{for-loop} 中使用字符 \verb|','| (comma) 分隔多条命令。
\item 阅读以下什么是 \ident{NULL},并尝试将数组 \ident{states} 的元素之一设为 
\ident{NULL},看看它会打印输出什么。 
\item 试试你可不可以在打印输出之前将数组 \ident{states} 的元素赋值到数组 
\ident{argv} 中,然后再反过来试一下。
\end{enumerate}


