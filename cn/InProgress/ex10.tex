\chapter{习题10: 字符串数组,循环}

You can make an array of various types, and have the idea down that a
"string" and an "array of bytes" are the same thing.  The next thing is
to take this one step further and do an array that has strings in it.
We'll also introduce your first looping construct, the \ident{for-loop}
to help print out this new data structure.
你可以用多种数据类型组成一个数组,要记得“字符串”和“字节数组”就是一个东西。下一步就是将字符串放到数组里。这里还会为你介绍第一个循环结构,For循环 ( \ident{for-loop} ) 会帮助你打印出这个新的数据结构。

The fun part of this is that there's been an array of strings hiding in
your programs for a while now, the \ident{char *argv[]} in the \ident{main}
function arguments.  Here's code that will print out any command line
arguments you pass it:
有个有趣的地方是,已经有个字符串数组在你的程序中躲藏了一段时间了, 那就是在 \ident{main} 函数形参中的 \ident{char *argv[]} 。这里是一段代码,它会打印出每一条你传递了的命令行:

\begin{code}{ex10.c}
<< d['code/ex10.c|pyg|l'] >>
\end{code}

The format of a \ident{for-loop} is this:
\ident{for-loop} 的格式是这样的:

\begin{Verbatim}
    for(INITIALIZER; TEST; INCREMENTER) {
        CODE;
    }
\end{Verbatim}

Here's how the \ident{for-loop} works:
这里将告诉你 \ident{for-loop} 是如何工作的:

\begin{enumerate}
\item The \ident{INITIALIZER} is code that is run to setup the loop, in this
    case \verb|i = 0|.
\item Next the \ident{TEST} boolean expression is checked, and if it's false (0)
    nothing is done.
\item The \ident{CODE} runs, does whatever it does.
\item After the \ident{CODE} runs, the \ident{INCREMENTER} part is run, usually
    incrementing something, like in \verb|i++|.
\item And it continues again with Step 2 until the \ident{TEST} is false (0).
\end{enumerate}

\begin{enumerate}
\item \ident{INITIALIZER} 是一个帮助你准备循环的代码,在这个情况下 \verb|i = 0| 。
\item 下一步中, \ident{TEST} 检查布尔型表达式,如果它的值是假 false (0),那么什么都不会发生。
\item 运行代码 \ident{CODE} 。
\item 运行代码之后, \ident{INCREMENTER}部分就会被运行,通常都是将某个变量递增,比如 \verb|i++| 。
\item 然后代码会重新运行, 直到第二步 \ident{TEST} 中出现 false (0) 。
\end{enumerate}

This \ident{for-loop} is going through the command line arguments 
using \ident{argc} and \ident{argv} like this:
这个 \ident{for-loop} 将用 \ident{argc} 和 \ident{argv} 通过所有命令行参数:

\begin{enumerate}
\item The OS passes each command line argument as a string in the \ident{argv}
    array.  The program's name (./ex10) is at 0, with the rest coming after it.
\item The OS also sets \ident{argc} to the number of arguments in the \ident{argv}
    array so you can process them without going past the end.
\item The \ident{for-loop} sets up with \verb|i = 1| in the initializer.
\item It then tests that \ident{i} is less than \ident{argc} with the
    test \verb|i < argc|. Since initially $0 < 1$ it will pass.
\item It then runs the code which just prints out the \ident{i} and 
    uses \ident{i} to index into \ident{argv}.
\item The incrementer is then run using the \verb|i++| syntax, which is
    a handy way of writing \verb|i = i + 1|.
\item This then repeats until \verb|i < argc| is finally false (0) when
    the loop exits and the program continues on.
\end{enumerate}

\begin{enumerate}
\item OS将,像一个字符串在数组 \ident{argv} 中一样,传递每一条命令行参数。程序名称 (./ex10) 在第零位。
\item The OS also sets \ident{argc} to the number of arguments in the 	
	\ident{argv} array so you can process them without going past the end.
\item 在初始值中 \ident{for-loop} 设定 \verb|i = 1| 。
\item 之后它将测试 \ident{i} 是否小于 \ident{argc} (\verb|i < argc|)。初始中,$0 < 1$ 是会通过测试的。
\item It then runs the code which just prints out the \ident{i} and 
    uses \ident{i} to index into \ident{argv}.
\item 之后递增器将通过语法 \verb|i++| 运行。 这是一种写出 \verb|i = i + 1| 的便捷方式。
\item 然后它就会重复循环中的代码,直到 \verb|i < argc| 不成立,然后循环结束,余下的程序继续运行。
\end{enumerate}


\section{你应该看到的结果}

To play with this program you have to run it two ways.  The first way is to
pass in some command line arguments so that \ident{argc} and \ident{argv}
get set.  The second is to run it with no arguments so you can see that
the first \ident{for-loop} doesn't run since \verb|i < argc| will be false.
跟这个程序打交道你需要用两种方式运行它。第一种方式,通过传递一些命令行参数设定 \ident{argc} 和 \ident{argv} 。第二种方式,不使用参数运行,这样的话这个 \ident{for-loop} 将不会运行,因为一开始 \verb|i < argc| 的值就是假。

\begin{code}{ex10 output}
\begin{lstlisting}
<< d['code/ex10.out|dexy'] >>
\end{lstlisting}
\end{code}

\subsection{弄明白字符串数组}

From this you should be able to figure out that in C you make an "array of
strings" by combining the \verb|char *str = "blah"| syntax with the
\verb|char str[] = {'b','l','a','h'}| syntax to construct a 2-dimensional
array.  The syntax \verb|char *states[] = {...}| on line 13 is this
2-dimension combination, with each string being one element, and each
character in the string being another.
这里你将弄明白,在C中做出一个“字符串数组”,是通过结合 语法\verb|char *str = "blah"| 和语法 \verb|char str[] = {'b','l','a','h'}| 来构建二维数组的。这个二维组合 (第13行的语法 \verb|char *states[] = {...}| ) ,每一个字符串是一个元素,而每一个字符串中的字符是另一个元素。

Confusing? The concept of multiple dimensions is something most
people never think about so what you should do is build this
array of strings on paper:
头晕?很多人都没有想过多重维度的概念,所以你现在应该试着在纸上 build 这个字符串数组:

\begin{enumerate}
\item Make a grid with the index of each \emph{string} on the left.
\item Then put the index of each \emph{character} on the top.
\item Then, fill in the squares in the middle with what single character
    goes in that cell.
\item Once you have the grid, trace through the code manually
    using this grid of paper.
\end{enumerate}

\begin{enumerate}
\item 在这张纸上做一个网格,最左列放入每一个 \emph{字符串} 的指针数。
\item 最顶行放入每一个 \emph{字符} 的指针数。
\item 然后,将字符按照序列填入这个表格中。
\item 你得到这个网格之后,就可以通过它来手动追踪代码了。
\end{enumerate}

Another way to figure it out is to build the same structure
in a programming language you are more familiar with like Python or
Ruby.
另一个方法是通过使用你更熟悉的程序语言 (比如 Python 或者 Ruby) build 一个相同的结构。

\section{让程序出错}

\begin{enumerate}
\item Take your favorite other language, and use it to run this program, but
    with as many command line arguments as possible.  See if you can bust it
    by giving it way too many arguments.
\item Initialize \ident{i} to 0 and see what that does.  Do you have to adjust
    \ident{argc} as well or does it just work?  Why does 0-based indexing work
    here?
\item Set \ident{num\_states} wrong so that it's a higher value and see what
    it does.
\end{enumerate}

\begin{enumerate}
\item 选一个你喜欢的语言,用它来运行这个程序,但是你要试着尽多的使用命令行参数。看看你能不能通过给出太多的形参把这程序给搅黄了。
\item 初始化 \ident{i} 为 0 看看会发生什么。 你需要同时调整 \ident{argc} 吗?还是说它可以直接运行?为什么这里可以用 0-based 的指针法呢?
\item 把 \ident{num\_states} 设置得更高导致错误,看看会发生什么。
\end{enumerate}

\section{加分习题}

\begin{enumerate}
\item Figure out what kind of code you can put into the parts of a \ident{for-loop}.
\item Look up how to use the \verb|','| (comma) character to separate multiple
    statements in the parts of the \ident{for-loop}.
\item Read what a \ident{NULL} is and try to use it in one of the elements of the
    \ident{states} array to see what it'll print.
\item See if you can assign an element from the \ident{states} array to the
    \ident{argv} array before printing both.  Try the inverse.
\end{enumerate}


