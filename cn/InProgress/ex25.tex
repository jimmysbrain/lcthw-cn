\chapter{习题 25: 变参函数}

在 C 语言里,你能创建类似 \func{printf} 和 \func{scanf} 这样的变参函数。
这些变参函数使用到了 \file{stdarg.h} 头文件,变参函数能给你的库文件创建良好的接口。
它们对于某些“构建”函数,格式化函数以及任何要取得不確定参数的函数都很方便。
//builder function 这个是什么?不知道怎么翻译才好。 

懂得“变参函数”对于写 C 程序\emph{不}是必须的。回想我写程序的这些年里,
我只用过大概20多次。无论如何,理解变参函数如何运作将会有助于你调试程序
和加深对计算机的理解程度。

\begin{code}{ex25.c}
<< d['code/ex25.c|pyg|l'] >>
\end{code}

这段程序除开我写了我自己的 \func{scanf} 风格的函数用来处理字符串以外和上一个习题没什么不同。
主函数对你来说应该很清晰明了,另外两个函数 \func{read\_string} 和 \func{read\_int} 
也没什么大的新意。

\func{read\_scan} 是一个变参函数,它做了同 \func{scanf} 一样的事,操作 \ident{va\_list}
数据,并且它提供宏和函数。下面是它如何工作的:

\begin{enumerate}
\item I set as the last parameter of the function the keyword \verb|...|
\item I set as the last parameter of the function the keyword \verb|...|
    which indicates to C that this function will take any number of arguments
    after the \ident{fmt} argument.  I could put many other arguments before
    this, but I can't put anymore after this.
\item After setting up some variables, I create a \ident{va\_list} variable
    and initialize it with \ident{va\_start}.  This configures the gear
    in \file{stdarg.h} that handles variable arguments.
\item I then use a \ident{for-loop} to loop through the format string
    \ident{fmt} and process the same kind of formats that \func{scanf}
    has, but much simpler.  I just have integers, characters, and strings.
\item When I hit a format, I use the \ident{switch-statement} to figure out
    what to do.
\item Now, to \emph{get} a variable from the \ident{va\_list argp} I use the
    macro \ident{va\_arg(argp, TYPE)} where TYPE is the exact type of what
    I will assign this function parameter to.  The downside to this design
    is you're flying blind, so if you don't have enough parameters then
    oh well, you'll most likely crash.
\item The interesting difference from \func{scanf} is I'm assuming that people
    want \func{read\_scan} to create the strings it reads when it hits a
    \ident{'s'} format sequence.  When you give this sequence, the function
    takes two parameters off the \ident{va\_list argp} stack: the max function
    size to read, and the output character string pointer.  Using that information
    it just runs \func{read\_string} to do the real work.
\item This makes \func{read\_scan} more consistent than \func{scanf} since you
    \emph{always} give an address-of \verb|\&| on variables to have them set
    appropriately.
\item Finally, if it encounters a character that's not in the format, it just reads
    one char to skip it.  It doesn't care what that char is, just that it should
    skip it.
\end{enumerate}


\section{你应该看到的结果}

当你运行这个程序,输出结果应类似以下:

\begin{code}{ex25.out}
\begin{lstlisting}
<< d['code/ex25.out'] >>
\end{lstlisting}
\end{code}

\section{让程序出错}

This program should be more robust against buffer overflows, but it doesn't
handle the formatted input as well as \func{scanf}.  To try breaking this,
change the code that you forget to pass in the initial size for '\%s' formats.
Try also giving it more data than \ident{MAX\_DATA}, and then see how not
using \func{calloc} in \func{read\_string} changes how it works.  Finally,
there's a problem that fgets eats the newlines, so try to fix that using
\func{fgetc} but leave out the '\\0' that ends the string.

\section{ 加分习题}

\begin{enumerate}
\item Make double and triple sure that you know what each of the \ident{out\_}
    variables are doing.  Most important is \ident{out\_string} and how it's
    a pointer to a pointer, so getting when you're setting the pointer vs. the
    contents is important.  Break down each of the 
\item Write a similar function to \func{printf} that uses the varargs system
    and rewrite \func{main} to use it.
\item As usual, read the man page on all of this so you know what it does
    on your platform.  Some platforms will use macros and others use
    functions, and some have these do nothing.  It all depends on the 
    compiler and the platform you use.
\end{enumerate}

