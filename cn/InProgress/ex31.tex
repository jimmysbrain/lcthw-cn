\chapter{Exercise 31: Debugging Code}

I've already taught you about my awesome debug macros and you've been using
them.  When I debug code I use the \ident{debug()} macro almost exclusively to
analyze what's going on and track down the problem.  In this exercise I'm going
to teach you the basics of using gdb to inspect a simple program that runs and
doesn't exit.  You'll learn how to use gdb to attach to a running process, stop
it, and see what's happening.  After that I'll give you some little tips and
tricks that you can use with gdb.
我们已经学习过如何使用超强悍 debug macros。当我们调试代码的时候专门使用
\ident{debug()}宏来跟踪代码出错的地方。这一章节,我们将会学习到一些基
本的 gdb 调试命令,用它来调试一个无法正常退出的程序。而且,您将会掌握如何使
用 gdb 去调试一个已经运行的进程。最后,我会传授您一些如何更好使用 gdb 的
技巧。

\section{Debug Printing Vs. GDB Vs. Valgrind}

I approach debugging primarily with a "scientific method" style, where I come
up with possible causes and then rule them out or prove they cause the defect.
The problem many programmers have though is their panic and rush to solve a bug
makes them feel like this approach will "slow them down".  In their rush to
solve they fail to notice that they're really just flailing around and
gathering no useful information.  I find that logging (debug printing) forces
me to solve a bug scientifically and it's also just easier to gather
information in more situations.
我们将用科学的方法去探讨如何调试代码。我们尽可能想出出错原因,排除它
或者证明是什么引起错误。很多程序员都有过这样的想法,担心代码出错或者急于找
到代码中的错误会“影响开发进度”。如果急忙去排错,只会被收集到的错误信息耍得团团转。
我们有更科学的方法解决错误--记录日志(打印调试信息),这样收集来的信息
更科学更简单。

In addition to that, I also have these reasons for using debug printing as my
primary debugging tool:
除此之外,还有以下几点原因把打印调试信息做为我的首要调试方法。
\begin{enumerate}
\item You see an entire tracing of a program's execution with debug printing of variables which lets you
    track how things are going wrong.  With gdb you have to place watch and debug statements all over
    for every thing you want and it's difficult to get a solid trace of the execution.
\item The debug prints can stay in the code, and when you need them you can recompile and they come back.
    With gdb you have to configure the same information uniquely for every defect you have to hunt down.
\item It's easier to turn on debug logging on a server that's not working right and then inspect the logs while
    it runs to see what's going on.  System administrators know how to handle logging, they don't know how
    to use gdb.
\item Printing things is just easier. Debuggers are always obtuse and weird with their own quirky interface and
    inconsistencies.  There's nothing complicated about \verb|debug("Yo, dis right? %d", my_stuff);|.
\item Writing debug prints to find a defect forces you to actually analyze the code and use the scientific method.
    You can think of a debug usage as, "I hypothesize that the code is broken here."  Then when you run it
    you get your hypothesis tested and if it's not broken then you can move to another part where it could be.
    This may seem like it takes longer, but it's actually faster because you go through a process of "differential
    diagnosis" and rule out possible causes until you find the real one.
\item Debug printing works better with unit testing.  You can actually just compile the debugs in all the time
    while you work, and when a unit test explodes just go look at the logs any time.  With gdb you'd have to
    rerun the unit test under gdb and then trace through it to see what's going on.
\item With valgrind you get the equivalent of debug prints for many memory related errors, so you don't need to
    use something like gdb to find those defects anymore.
\end{enumerate}

\begin{enumerate}
\item 打印调试信息在整个程序运行中可以让您发现变量运行到哪一步出错。如
  果使用 gdb 你需要给变量设观察点(watch)调试所有有关语句,想很好定位出
  错地方是很困难的。
\item 打印调试信息可以保留在代码里。在需要时把他们编译出来就行了。
  如果使用 gdb 你需要给相同的信息使用不同设置,直到把问题一个一个找出
  来。
\item 程序在一台服务器中运行异常,我们可以简单开启程序打印调试信息功能查看发生
  了什么情况。系统管理员更喜欢去分析程序日志,而不是使用 gdb 排错。
\item 打印信息很简单。调试器古怪的界面总会与它给出的信息不太协调。还不
  如来一句 \verb|debug("Yo, dis right? %d", my_stuff)| 简单明了。
\end{enumerater}


Despite all these reasons that I rely on \ident{debug} over \program{gdb}, I
still use \program{gdb} in a few situations and I think you should have any
tool that helps you get your work done.  Sometimes, you just have to connect to
a broken program and poke around.  Or, maybe you've got a server that's
crashing and you can only get at core files to see why.  In these and a few
other cases, gdb is the way to go, and it's always good to have as many tools
as possible to help solve problems.

I then break down when I use gdb vs. valgrind vs. debug printing like this:

\begin{enumerate}
\item Valgrind is used to catch all memory errors.  I use gdb if valgrind is having problems or if using
    valgrind would slow the program down too much.
\item Print with debug to diagnose and fix defects related to logic or usage.  This amounts to about 90\% of the
    defects after you start using Valgrind.
\item Use gdb for the remaining "mystery weird stuff" or emergency situations to gather information.  If Valgrind isn't turning anything
    up and I can't even print out the information I need, then I bust out gdb and start poking around.  My use of
    gdb in this case is entirely to gather information.  Once I have an idea of what's going on I go back to writing
    a unit test to cause the defect, and then do print statements to find out why.
\end{enumerate}


\section{A Debugging Strategy}

This process will actually work with any debugging technique you're going to
use, whether that's Valgrind, debug printing, or using a debugger.  I'm going
to describe it in terms of using \program{gdb} since it seems people skip this
process the most when using debuggers, but use this for every bug until you
only need it on the very difficult ones.

\begin{enumerate}
\item Start a little text file called \file{notes.txt} and use it as a kind of
    "lab notes" for ideas, bugs, problems, etc.
\item Before you use \program{gdb}, write out the bug you're going to fix
    and what could be causing it.
\item For each cause, write out the files and functions where you think 
    the cause is coming from, or just write that you don't know.
\item Now start \program{gdb} and pick the first possible cause with good
    file:function possibles and set breakpoints there.
\item Use \program{gdb} to then run the program and confirm if that is the
    cause.  The best way is to see if you can use the \ident{set} command
    to either fix the program easily or cause the error immediately.
\item If this isn't the cause, then mark in the \file{notes.txt} that it
    wasn't and why.  Move on to the next possible cause that's easiest
    to debug, and keep adding information you gather.
\end{enumerate}

In case you haven't noticed, this is basically the scientific method.  You
write down a set of hypotheses, then you use debugging to prove or disprove
them.  This gives you insight into more possible causes and then eventually you
find it. This process helps you avoid going over the same possible causes
repeatedly even though you've found they aren't possible.

You can also do this with debug printing, the only difference is you actually
write out your hypotheses in the source code where you think the problem is
instead of the \file{notes.txt}.  In a way, debug printing forces you to tackle
bugs scientifically since you have to write out hypotheses as print statements.


\section{Using GDB}

The program I'll debug in this exercise is just a while-loop that doesn't
terminate correctly.  I'm putting a small \ident{usleep} call in it so that
there's something interesting to troll through as well.
接下来我们要调试一个无法正确终止的 while 语句。while 语句中加入了
\ident{usleep} 函数,这样调试起来更加方便。

\begin{code}{ex31.c}
<< d['code/ex31.c|pyg|l'] >>
\end{code}

Compile this like normal and then start it under \program{gdb} like this:  \verb|gdb ./ex31|
正常编译它,然后使用 \program{gdb} 启动。像这样:\verb|gdb ./ex31|
Once it's running I want you to play around with these \program{gdb} commands
to see what they do and how to use them.
调试开始时,我们将用到下列 \program{gdb} 命令。

\begin{description}
\item[help COMMAND] Get a short help with COMMAND.
\item[break file.c:(line|function)] Sets a break point where you want to pause execution.  You can give lines or function names to break at after the file.
\item[run ARGS] Runs the program, using the ARGS as arguments to the program.
\item[cont] Continues execution until a new breakpoint or error.
\item[step] Step through the code, but move \emph{into functions}.  Use this to 
    trace into a function and see what it's doing.
\item[next] Just like \ident{step}, but go \emph{over functions} by just running them.
\item[backtrace (or bt)] Does a "backtrace", which dumps the trace of function
    calls leading to the current point in the program. Very useful for figuring
    out how you got there, since it also prints the parameters that were passed
    to each function.  It's also similar to what Valgrind reports when you have
    a memory error.
\item[set var X = Y] Set variable X equal to Y.
\item[print X] Prints out the value of X, and you can usually use C syntax to access
    the values of pointers and contents of structs.
\item[ENTER] The ENTER key just repeats the last command.
\item[quit] Exits \program{gdb}
\end{description}

\begin{description}
\item[help COMMAND] 显示命令使用方法。
\item[break file.c(行号|函数名)] 给程序需要暂停的地方设置一个断点。你
  可以行号或者函数名设置一个断点。
\item[run 参数] 运行程序。run 后面加上参数可以把参数传递给调试的程序。
\item[cont] 让程序一直运行,直到遇到断点或错误。
\item[step] 程序执行下一步。但是 \emph{进入函数内部}。用他可查看函数内
  部动作状况。
\item[next] 与 \ident{step} 功能差不多。但是 \emph{跳过当前函数},直接使用
  函数。
\item[backtrace (bt)] 显示栈帧,查看函数在程序中运行的地址。可以查看停
  留在程序的什么地方,同样可以列出传给函数的参数。与 Valgrind 内存错误
  报告差不多。
\item[set var X = Y] 重新赋值 X 变量为 Y 。
\item[print X] 打印 X 变量的值。就像正常使用C语言语法一样,对变量的指
  针或者结构体取值。
\item[ENTER] 重复上一次命令。
\item[quit] 退出 \program{gdb}。
\end{description}




Those are the majority of commands I use with \program{gdb}.  Your job is to
now play with these and \program{ex31} so you can get familiar with the output.

Once you're familiar with \program{gdb} you'll want to play with it some more.
Try using it on more complicated programs like \program{devpkg} to see if you
can alter the program's execution or analyze what it's doing.


\section{Process Attaching}

The most useful thing about \program{gdb} is the ability to attach to a running program and
debug it right there.  When you have a crashing server or a GUI program, you can't
usually start it under \program{gdb} like you just did.  Instead, you have to start
it, hope it doesn't crash right away, then attach to it and set a breakpoint.  In
this part of the exercise I'll show you how to do that.

After you exit \program{gdb} I want you to restart \program{ex31} if you
stopped it, and then start another Terminal window so you can process attach to
it.  Process attaching is where you tell \program{gdb} to connect to a program
that's already running so you can inspect it live.  It stops the program and
then you can walk through it, and when you're done it'll continue just like
normal.

Here's a session of me doing it to \program{ex31}, stepping through it, then
fixing the while-loop to make it exit.

\begin{code}{ex31.sh-session}
<< d['code/ex31.sh-session|pyg|l'] >>
\end{code}

\begin{aside}{OSX Problems}
On OSX you may see a GUI prompt for the root password, and even after you
give it you still get an error from \program{gdb} saying "Unable to access task for process-id XXX: (os/kern) failure."  In that case stop both gdb and the
\program{ex31} program, then start over and it should work as long as you
successfully entered the root password.
\end{aside}

I'll walk through this session and explain what I did:

\begin{description}
\item[gdb:1] I use \program{ps} to find out what the process id is
    of the \program{ex31} I want to attach.
\item[gdb:5] I'm attaching using \verb|gdb ./ex31 PID| replacing PID with
    the process id I have.
\item[gdb:6-19] \program{gdb} prints out a bunch of information about it's
    license and then all the things it's reading. \footnote{Just in case you missed
    it that \program{gdb} really was the GNU debugger and just in case
    you didn't know it was doing all this stuff.}
\item[gdb:21] The program is attached and stopped at this point, so now I set
    a breakpoint at line 8 in the file with \ident{break}.
    I'm assuming that I'm already in the file I want to break when I do this.
\item[gdb:24] A better way to do a \ident{break}, is give \file{file.c:line} format
    so you can be sure you did the right location.  I do that in this \ident{break}.
\item[gdb:27] I use \ident{cont} to continue processing until I hit a breakpoint.
\item[gdb:30-31] The breakpoint is reached so \program{gdb} prints out variables
    I need to know about (\ident{argc} and \ident{argv}) and where it's stopped,
    then the line of code for the breakpoint.
\item[gdb:33-34] I use the abbreviation for \ident{print} "p" to print out the value
    of the \ident{i} variable.  It's 0.
\item[gdb:36] Continue again to see if \ident{i} changes.
\item[gdb:42] Print out \ident{i} again, and nope it's not changing.
\item[gdb:45-55] Use \ident{list} to see what the code is, and then I realize
    it's not exiting because I'm not incrementing \ident{i}.
\item[gdb:57] Confirm my hypothesis that \ident{i} needs to change by using the
    \ident{set} command to change it to be \verb|i = 200|.  This is one of the
    best features of \program{gdb} as it lets you "fix" a program really quick
    to see if you're right.
\item[gdb:59] Print out \ident{i} just to make sure it changed.
\item[gdb:62] Use \ident{next} to move to the next piece of code, and I see that
    the breakpoint at \file{ex31.c:12} is hit, so that means the while-loop
    exited.  My hypothesis is correct, I need to make \ident{i} change.
\item[gdb:67] Use \ident{cont} to continue and the program exits like normal.
\item[gdb:71] I finally use \ident{quit} to get out of \program{gdb}.
\end{description}

\section{GDB Tricks}

Here's a list of simple tricks you can do with GDB:
这里有一些在 gdb 调试时可以使用的小技巧。
\begin{description}
\item[gdb --args] Normally \program{gdb} takes arguments you give it
    and assumes they are for itself.  Using \program{--args} passes them to 
    the program.
\item[thread apply all bt] Dumps a backtrace for \emph{all} threads.  Very useful.
\item[gdb --batch --ex r --ex bt --ex q --args] Runs the program so that, if it
    bombs you get a backtrace.
\item[?] Got one? Leave it in the comments.
\end{description}

\begin{description}
\item[gdb --args] 通常情況下,启动带参数的 \program{gdb} 会把所有的参数当成
  自己的。我们只需在 gdb 后面加上 \program{--args} 就可以把参数
  传给要调试的程序。(如:gdb --args ex31 ``hello world'')
\item[thread apply all bt] 多线程调试时,使用 backtrace 可以显示
  \emph{所有}线程栈帧信息。很有用。
\item[gdb --batch --ex r --ex bt --ex q --args] 如果运行一个程序异常退
  出,用这个方法可以很快定位出错地方。并显示栈帧信息。
\item[?] (待译)
\end{description}

\section{Extra Credit}

\begin{enumerate}
\item Find a graphical debugger and compare using it to raw \program{gdb}.
    These are useful when the program you're looking at is local, but they
    are pointless if you have to debug a program on a server.
\item You can enable "core dumps" on your OS, and when a program crashes
    you'll get a core file.  This core file is like a post-mortem of
    the program so you can load up what happened right at the crash
    and see what caused it.  Change \file{ex31.c} so that it crashes
    after a few iterations, then try to get a core dump and analyze it.
\end{enumerate}

\begin{enumerate}
\item 找个图形界面的调试器与命令行的 \program{gdb} 作比较,熟练使用他
  们。当远程调试服务器上的程序时,图形界面的调试器看起来更加本地化,更
  容易上手。
\item 启用系统中的 “内核转储(core dump)”\footnote{译注:大多数 linux
    发行版本默认情况下关闭了内核转储功能。用 ulimit -c 命令可以查看当前的内核转储
    功能是否有效。如 ulimit -c 1073741824 可以限制内核转储大小为 1GB。}。
  程序崩溃时可以得到一个内核转储文件。调试时加载此文件,可以看到程序
  崩溃前运行状态,是什么引起程序崩溃,就如同给程序验尸一样。尝试修改
  \file{ex31.c} 文件的代码,让它迭代至崩溃。然后分析下得到的内核转储文
  件。
\end{enumerate}