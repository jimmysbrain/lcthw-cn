\chapter{习题 4: 介绍 Valgrind}

是时候学习另外一个工具 \program{Valgrind} 了,它将伴随你学习 C 语言的整个过程。我现在介绍 \program{Valgrind} 给你,因为从现在起,在“让程序出错”这个小节里,每个练习你都将用到它。\program{Valgrind} 运行你的程序,然后报告你犯下的所有致命错误。它是一个很棒的自由软件,我经常在我写 C 代码的时候用到它。

还记得在上一个习题里,我让你修改你的代码移除了 \ident{printf} 函数的参数吗?它打印出了一些补寻常的结果,但我没有告诉你,为什么它会打印出这些结果。在这个习题里,我们将用 \program{Valgrind} 来探个究竟。

\begin{aside}{What's With All The Tools}
只经过几章,我们就学习了本书所需要的所有工具,而只学写了一点点代码。这是因为这本书的大部分读者不熟悉编译语言,当然也不知道自动化处理和有用的工具。让你马上接触 \ident{make}  和 \program{Valgrind} ,我就能用他们更快的教会你 C 语言并帮助你找到你在程序中犯的错误。

在这章练习后,一段时间内,我们将不会介绍其他任何工具,大部分将是代码和语法。但我们还将会学习一些工具,用来查看程序如何运行和帮我们理解一些常见错误和问题 。

\end{aside}

\section{安装 Valgrind}

 你能通过操作系统的包管理器来安装 \program{Valgrind},但我要你学习如何从源代码安装程序。它包括以下几个步骤:

\begin{enumerate}
\item 下载源代码包。
\item 解压文件到你的电脑。
\item 运行 \program{./configure} 设置配置。
\item 运行 \program{make} 创建程序, 就像你以前做过的那样。
\item 运行 \program{sudo make install} 把程序安装到你的电脑。
\end{enumerate}

下面是我流程的一个脚本,我要你试着重复做一次。
try to replicate:

\begin{code}{ex4.sh}
<< d['code/ex4.sh|pyg|l'] >>
\end{code}

按以上流程,但很明显,要根据新版的 Valgrind 更新流程。如果它创建失败,那么查出为什么出错。

\section{Using Valgrind}

Using \program{Valgrind} is easy, you just run \verb|valgrind theprogram| and
it runs your program, then prints out all the errors your program made while it
was running.  In this exercise we'll break down one of the error outputs and
you can get an instant crash course in "Valgrind hell".  Then we'll fix the
program.

First, here's a purposefully broken version of the \file{ex3.c} code
for you to build, now called \file{ex4.c}.  For practice, type it
in again:

\begin{code}{ex4.c}
<< d['code/ex4.c|pyg|l'] >>
\end{code}

You'll see it's the same except I've made two classic mistakes:

\begin{enumerate}
\item I've failed to initialize the \ident{height} variable.
\item I've forgot to give the first \ident{printf} the \ident{age} variable.
\end{enumerate}

\section{What You Should See}

Now we will build this just like normal, but instead of running it
directly, we'll run it with \program{Valgrind} (see Source: "Building and running ex4.c with Valgrind"):

\begin{Terminal}{Building and running ex4.c with Valgrind}
\begin{lstlisting}
<< d['code/ex4.out|dexy'] >>
\end{lstlisting}
\end{Terminal}

This one is huge because \program{Valgrind} is telling you exactly where
every problem in your program is.  Starting at the top here's what you're
reading, line by line (line numbers are on the left so you can follow):

\begin{description}
\item[1] You do the usual \verb|make ex4| and that builds it. Make sure the \ident{cc} command
    you see is the same and has the \verb|-g| option or your \program{Valgrind} output won't
    have line numbers.
\item[2-6] Notice that the compiler is also yelling at you about this source file and it
    warns you that you have "too few arguments for format".  That's where you 
    forgot to include the \ident{age} variable.
\item[7] Then you run your program using \verb|valgrind ./ex4|.
\item[8] Then \program{Valgrind} goes crazy and yells at you for:
    \begin{description}
        \item[14-18] On line \verb|main (ex4.c:11)| (read as "in the main function in
            file ex4.c at line 11) you have "Use of uninitialised value of size 8".
            You find this by looking at the error, then you see what's called a "stack trace"
            right under that.  The line to look at first (ex4.c:11) is the bottom one, 
            and if you don't see what's going wrong then you go up, so you'd try
            printf.c:35.  Typically it's the bottom most line that matters (in this case, on line 18).
        \item[20-24] Next error is yet another one on line ex4.c:11 in the main function. \program{Valgrind}
            hates this line.  This error says that some kind of if-statement or while-loop
            happened that was based on an uninitialized variable, in this case height.
        \item[25-35] The remaining errors are more of the same because the variable keeps getting
        used.
    \end{description}
\item[37-46] Finally the program exits and \program{Valgrind} tells you a summary of how bad
    your program is.
\end{description}

That is quite a lot of information to take in, but here's how you deal with it:

\begin{enumerate}
\item Whenever you run your C code and get it working, rerun it under \program{Valgrind}
    to check it.
\item For each error that you get, go to the source:line indicated and
    fix it.  You may have to search online for the error message to figure out
    what it means.
\item Once your program is "Valgrind pure" then it should be good, and you
    have probably learned something about how you write code.
\end{enumerate}

In this exercise I'm not expecting you to fully grasp \program{Valgrind} right
away, but instead get it installed and learn how to use it real quick so we
can apply it to all the later exercises.

\section{Extra Credit}

\begin{enumerate}
\item Fix this program using \program{Valgrind} and the compiler as your guide.
\item Read up on \program{Valgrind} on the internet.
\item Download other software and build it by hand. Try something you already
    use but never built for yourself.
\item Look at how the \program{Valgrind} source files are laid out in the
    source directory and read its Makefile.  Don't worry, none of that
    makes sense to me either.
\end{enumerate}

