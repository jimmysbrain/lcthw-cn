\chapter{Exercise 7: 更多变量和一些数学运算}

让我们来熟悉一下通过声明更多的\ident{整型}(int)变量, \ident{单精度浮点型}(float)变量, \ident{字符型}(char)变量和\ident{双精度浮点型}(double)变量可以做什么. 我们会在稍后的各种数学表达式中用到它们,接下来会交给你C语言中的基础数学.

\begin{code}{ex7.c}
<< d['code/ex7.c|pyg|l'] >>
\end{code}

这些就是要做的那些有意义的事情:

\begin{description}
\item[ex7.c:1-4] C程序的一般开始格式.
\item[ex7.c:5-6] 为一些冒牌的bug数据声明\ident{整型}(int)和\ident{双精度浮点型}(double)变量.
\item[ex7.c:8-9] 没什么新的知识点,就是把这俩变量打印出来.
\item[ex7.c:11] 用一种新的变量类型\ident{long}声明一个大数(huge number).
\item[ex7.c:12-13] 使用\verb|%ld|输出那个数, %ld只是在\verb|%d|前加了个修饰符. 增加的'l' (字母l)的意思是是"按长整型(long decimal)打印出这个数".
\item[ex7.c:15-17] 只是更多的声明并定义变量和打印输出.
\item[ex7.c:19-21] 修改一下描述你的bug和世界上别的bug的比例的值, 这是一个不准确的计算.  我们可以用修饰符\verb|%e|用科学计数法输出结果.
\item[ex7.c:24] 用\verb|'\0'|这样的语法赋值字符变量,就创建了一个空(nul byte)的字符.  这其实就是数字0.
\item[ex7.c:25] 用这个字符乘以bugs变量, 你可能已经想到了得0. 这就是有时你发现的一种不怎么优雅的hack(通过巧妙的设计实现某些功能)技巧.
\item[ex7.c:26-27] 输出那个结果, 注意到我使用\verb|%%| (两个百分号),所以我可以输出'\%' (百分号).
\item[ex7.c:28-30] \ident{main}函数的结尾.
\end{description}

This bit of source is entirely just an exercise, and demonstrates how
some math works.  At the end, it also demonstrates something you
see in C, but not in many other languages.  To C, a "character" is just an
integer.  It's a really small integer, but that's all it is.  This means
you can do math on them, and a lot of software does just that, for good
or bad.

This last bit is your first glance at how C gives you direct access to 
the machine.  We'll be exploring that more in later exercises.


\section{What You Should See}

As usual, here's what you should see for the output:

\begin{code}{ex7 output}
\begin{lstlisting}
<< d['code/ex7.out|dexy'] >>
\end{lstlisting}
\end{code}


\section{How To Break It}

Again, go through this and try breaking the \ident{printf} by passing in
the wrong arguments.  See what happens when you try to print out that
\ident{nul\_byte} variable too with \verb|%s| vs. \verb|%c|.  When you
break it, run it under \program{Valgrind} to see what it says about your
breaking attempts.

\section{Extra Credit}

\begin{enumerate}
\item Make the number you assign to \ident{universe\_of\_defects} various 
    sizes until you get a warning from the compiler.
\item What do these really huge numbers actually print out?
\item Change \ident{long} to \ident{unsigned long} and try to find 
    the number that makes that one too big.
\item Go search online to find out what \ident{unsigned} does.
\item Try to explain to yourself (before I do in the next exercise)
    why you can multiply a \ident{char} and an \ident{int}.
\end{enumerate}

