\chapter{习题8: 数组和大小}

上一个习题中你运用字符 \verb|'\0'| (nul) 做了些数学运算。

如果是其他语言的话就有可能会要有奇怪的事情发生了,因为它们用非常不同的方式对待字符串 ( "strings" ) 和比特数组 ( "byte arrays" )。 然而C语言则将它们在承认差异的情况下等同而待,仅仅是使用了不同的输出函数。

在解释这一点的意义重大之前,有几个概念一定要讲清楚: \ident{sizeof} 和数组。以下是我们即将探讨的代码:

\begin{code}{ex8.c}
<< d['code/ex8.c|pyg|l'] >>
\end{code}

在这段代码中我们将创建一些包含不同数据类型的数组。 由于数组对于C语言的运行相当重要,所以我们可以用很多种不同的方法来创建数组。 这里将暂且使用语法格式 \verb|type name[] = {initializer};| ,往后我们还会介绍更多其他方式。这个语法格式的意思就是, “我需要一个初始化到{..}的数组” ("I want an array of
type that is initialized to {..}." ), 当C接收到这个时将会: 

\begin{enumerate}
\item 关于数据类型,整数型 ( \ident{int} ) 是第一情况。
\item 关于 \verb|[]| ,并确定没有给出长度。
\item 关于初始值,发现你希望将以下五个值放入到数组 \verb|{10, 12, 13, 14, 20}|。
\item 在本机上腾出记可以保存这五个整数的存储空间(memory)。
\item 命名, \ident{areas} 并指明存储位置。
\end{enumerate}


在这种情况下 \ident{areas} 创建了容纳这五个整数型数据的数组。当在 \verb|char name[] = "Zed";|中运行时,它也是干了一样的活,除了它创建了一个有三个字符的数组并赋值到 \ident{name}。 我们创建出来的最终版数组是 \ident{full\_name}, 但一次一个字符的拼写方式却很恼人。对于C语言来说, \ident{name} 和 \ident{full\_name} 都等同的是创建字符数组的方式。

接下来我们将使用一个叫做 \ident{sizeof} 的关键词来询问C文件大小以 \emph{bytes} 为单位计算是多少。事实上C语言就是一门关于记忆大小,记忆位置,以及如何处理文记忆(pieces of memory)的语言。 更简单一点说,\ident{sizeof} 这个玩意可以让你知道你之前编辑的文件有多大。

这里可能会让人有些头晕,所以我们先运行下面这段代码再来解释。

\section{你应该看到的结果}

\begin{code}{ex8 output}
\begin{lstlisting}
<< d['code/ex8.out|dexy'] >>
\end{lstlisting}
\end{code}

现在你看到不同的 \ident{printf} 指令的输出了。到这里你也算对C如何工作有了一个简单的了解。 由于你的电脑可能会有不同大小的整数,你的输出有可能跟我的输出会完全不同,这里是我的输出:

\begin{description}
\item [5] 我的电脑以为 \ident{int} 有4个比特那么大. 你的电脑如果是32-bit的话,可能会有不同结果。
\item [6] 数组 \ident{areas} 里有5个整数, 我的电脑要求了20个比特来存储它。
\item [7] 如果我们用 \ident{areas} 的大小除以 \ident{int} 的大小,我们将得到元素数5。再看看代码,这正匹配了初始值中的设定。
\item [8] 我们来尝试一下使用数组直接得到 \verb|areas[0]| 和 \verb|areas[1]|, 这意味着C和Python/Ruby一样也是指针数从零位开始 ( "zero indexed" )。
\item [9-11] 为数组 \ident{name} 重复这个部分, 却发现这个数组的大小有些奇怪? \emph{4} 比特长?但 "Zed" 是三个字符。第四个字符是哪来的呢?
\item [12-13] 对 \ident{full\_name} 再次运行,并发现这次的值是正确的。 
\item [13] 最后我们打印输出 \ident{name} 和 \ident{full\_name} 以证明根据printf他们都是字符串 ( "strings" ).
\end{description}

请确保你可以弄懂这些代码,并且明白这些输出结果是如何与之匹配的。之后我们将在这个基础之上更多地探索关于数组和存储。

\section{让程序出错}

让这个程序出错是相当简单的,试一试这些:

\begin{enumerate}
\item 删除 \ident{full\_name} 尾部的 \verb|'\0'| 并重新运行程序。 也在 Valgrind 下运行。 现在,将 \ident{full\_name} 的定义到 \ident{main} 的顶部,并且要在 \ident{areas} 的前面。在 Valgrind 下运行几次看看你会不会得到新的系统错误。有些时候你可能好彩,一个系统错误都没有。
\item 改变它,去尝试打印输出 \verb|areas[10]| 而非 \verb|areas[0]| ,并看看  是怎样认为的。
\item 对 \ident{name} 和 \ident{full\_name} 都多尝试几个版本的尝试。
\end{enumerate}

\section{加分习题}

\begin{enumerate}
\item 使用 \verb|areas[0] = 100;|的类似语句对数组 \ident{areas} 中的元素进行赋值。
\item 尝试对 \ident{name} 和 \ident{full\_name} 的元素进行复制。
\item 尝试把 \ident{areas} 中的一个元素设置成 \ident{name} 中的一个字符.
\item 上网搜索:在不同CPU上整数的大小差异 (the different sizes used for integers on different CPUs).
\end{enumerate}

