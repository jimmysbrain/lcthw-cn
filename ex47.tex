\chapter{Exercise 47: A Fast URL Router}

I'm going to now show you how I use the \ident{TSTree} to do fast 
URL routing in web servers I've written.  This works for simple 
URL routing you might use at the edge of an application, not really
for the more complex (and sometimes unecessary) routing found in
many web application frameworks.

To play with routing I'm going to make a little command line tool I'm calling
\program{urlor} that reads a simple file of routes, and then prompts the user
to enter in URLs to look up.

\begin{code}{bin/urlor.c}
<< d['code/liblcthw/bin/urlor.c|pyg|l'] >>
\end{code}

I'll then make a simple file with some fake routes to play with:

\begin{code}{urls.txt}
\begin{Verbatim}
<< d['code/ex47_urls.txt'] >>
\end{Verbatim}
\end{code}

\section{What You Should See}

Once you have \program{urlor} working and a routes file, you can try it out:

\begin{code}{Working With urlor}
<< d['code/ex47.sh-session|pyg|l'] >>
\end{code}

You can see that the routing system first tries an exact match, and then if it
can't find one it will give a prefix match.  This is mostly to try out the 
difference between the two.  Depending on the semantics of your URLs you may
want to always match exactly, always to prefixes, or do both and pick the "best"
one.

\section{How To Improve It}

URLs are weird because people want them to magically handle all
of the insane things their web applications do, even if that's not very logical.
In this  simple demonstration of how to use the \ident{TSTree} to do routing,
it has some flaws that people wouldn't be able to articulate.  For example,
it will match \file{/al} to \ident{Album}, which generall isn't what they want.
They want \file{/album/*} to match \ident{Album} and \file{/al} to be a 404
error.

This isn't difficult to implement though, since you could change the prefix
algorithm to match any way you want.  If you change the matching algorithm to
find \emph{all} matching prefixes, and then pick the "best" one, you'll be
able to do it easily.  In this case, \file{/al} could match \ident{MainApp}
or \ident{Album}.  Take those results then do a little logic on which is "best".

Another thing you can do in a real routing system is use the \ident{TSTree} to
finall possible matches, but that these matches are a small set of patterns
to check.  In many web applications there's a list of regex that have to be
matched against URLs on each request.  Running all the regex can be time
consuming, so you can use a \ident{TSTree} to find all the possible ones
by their prefixes.  Then you narrow the patterns to try down to a few
very quickly.

Using this method, your URLs will match exactly since you are actually running
real regex patterns, and they'll match much faster since you're finding them
by possible prefixes.

This kind of algorithm also works for anything else that needs to have flexible
user-visible routing mechanisms.  Domain names, IP address, registries and directories,
files, or URLs.

\section{Extra Credit}

\begin{enumerate}
\item Instead of just storing the string for the handler, create an actual engine that uses an
    \ident{Handler} struct to store the application.  The struct would store the URL it is
    attached to, the name, and anything else you'd need to make an actual routing system.
\item Instead of mapping URLs to arbitrary names, map them to .so files and use the \ident{dlopen}
    system to load handlers on the fly and call callbacks they contain.  Put these callbacks
    in your \ident{Handler} struct and then you have yourself a fully dynamic callback 
    handler system in C.
\end{enumerate}

