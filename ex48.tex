\chapter{Exercise 48: Ternary Search Tree}

My favorite algorithm, a TST creates a binary tree of the characters in a set
of strings, then sets the terminating character of each string to a piece of
data.  This effectively creates a kind of "suffix array hash map" or "character
binary tree".  It's very useful in quickly matching a given string to a set of
strings, such as in web application routing.

I show a simplified implementation used in Mongrel and Mongrel2 and how it works.


\section{What You Should See}


\section{How To Break It}


\section{Extra Credit}



